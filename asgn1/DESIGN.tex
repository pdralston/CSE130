\documentclass[11pt,letterpaper]{article}

\usepackage[utf8]{inputenc}
\usepackage{amsmath}
\usepackage{amsfonts}
\usepackage{amssymb}
\usepackage{fancyhdr}
\usepackage{graphicx}
\usepackage{listings}% http://ctan.org/pkg/listings
\usepackage{xcolor}
\usepackage{hyperref}

\definecolor{codegreen}{rgb}{0,0.6,0}
\definecolor{codegray}{rgb}{0.5,0.5,0.5}
\definecolor{codepurple}{rgb}{0.58,0,0.82}
\definecolor{backcolour}{rgb}{0.95,0.95,0.92}
\lstdefinestyle{mystyle}{
    backgroundcolor=\color{backcolour},   
    commentstyle=\color{codegreen},
    keywordstyle=\color{magenta},
    numberstyle=\tiny\color{codegray},
    stringstyle=\color{codepurple},
    basicstyle=\ttfamily\footnotesize,
    breakatwhitespace=false,         
    breaklines=true,                 
    captionpos=b,                    
    keepspaces=true,                 
    numbers=left,                    
    numbersep=5pt,                  
    showspaces=false,                
    showstringspaces=false,
    showtabs=false,
    tabsize=2
}

\hypersetup{
    colorlinks=true,
    linkcolor=blue,
    filecolor=magenta,      
    urlcolor=cyan,
}

\lstset{style=mystyle}

\pagestyle{fancy}
\fancyhf{}
\lhead{\footnotesize{Design Document: rpcserver}}
\rhead{\footnotesize{Perry David Ralston Jr. CruzID: pdralsto}}
\cfoot{\footnotesize{Page \thepage}}


\title{Design Document: rpcserver}
\author{Perry David Ralston Jr.\\ CruzID: pdralsto}

\begin{document}
\maketitle
\thispagestyle{fancy}
\section{Goals}

This RPC Server will handle requests from a client to perform simple arithmetic: addition, subtraction, and multiplication (64 bit width) and basic file operations: read, write, create, and filesize query. All messages to the server are assumed to be in network byte order and all responses from the server will be formatted the same.
\section{Initialization}

Server initialization is handled at the command line by running rpcserver with the arg $<$host$\_$name$>$:$<$port$>$. Using code retrieved from \href{https://canvas.ucsc.edu/courses/36179/pages/setting-up-sockets-for-basic-client-slash-server-stream-communication}{CSE130's Canvas Page}. The command line arg is split at the ':' and the hostname and port are placed in the appropriate positions. The hostname must be $\leq$ 1 kB in size and the server will crash out otherwise. Once the socket has been created and bound, rpcserver runs an infinte loop listening on the address and port specified at the command line.
\subsection{Handling Requests and Sending Responses}

Requests from the client are assumed to be in network byte order. Responses will be formatted in network byte order prior to sending responses "over-the-wire". The subroutines shown below detail the process for converting to and from network byte order. This server makes no assumptions about the rate at which all of the data is received from the client. Messages from the client are read into a bounded$\_$buffer object for processing.\\

\begin{minipage}{\linewidth}
\lstinputlisting[language=Python, frame=single]{convert_to_nbo}
\vspace{-.5cm}\begin{center}
\footnotesize{\textbf{Convert to Network Byte Order}}
\end{center}
\end{minipage}\\\\

\begin{minipage}{\linewidth}
\lstinputlisting[language=Python, frame=single]{convert_from_nbo}
\vspace{-.5cm}\begin{center}
\footnotesize{\textbf{Convert from Network Byte Order}}
\end{center}
\end{minipage}

\subsection{Bounded$\_$Buffer Class}

The Bounded$\_$Buffer class is responsible for maintaining, filling, and flushing the bounded buffer used to store messages to and from the client for internal processing. It has private members for three uint8$\_$t pointers that maintain the root, start and end of the buffer and public functions empty, fill, flush, getByte, getBytes, pushByte, and pushBytes. The public functions are detailed below:

\lstinputlisting[language=Python, frame=sing1le]{bounded_buffer}
\vspace{-.5cm}\begin{center}
\footnotesize{\textbf{Bounded$\_$Buffer Public Functions}}
\end{center}

\subsection{Resolving Arguments and Calling Functions}

Once the request has been parsed, the corresponding function call is made. If no corresponding function call can be found then response header containing \textcolor{red}{EBADRQC} is sent back to the client. Arguments to the corresponding function are parsed from the data portion of the request.

\pagebreak  
\subsection{Supported Functions}
\paragraph*{Math Functions} Add, Subtract, and Multiply are supported\par
\vspace{.5cm}\begin{minipage}{\linewidth}
\subparagraph*{add} Add two numbers, A and B, together returning the value. If overflow would occur set err$\_$code to EINVAL(22)
\lstinputlisting[language=Python, frame=single]{add}
\end{minipage}\\\\

\begin{minipage}{\linewidth}
\subparagraph*{subtract} Subtract B from A, returning the value. If overflow would occur set err$\_$code to EINVAL(22)
\lstinputlisting[language=Python, frame=single]{subtract}
\end{minipage}\\\\

\begin{minipage}{\linewidth}
\subparagraph*{multiply}Add two numbers, A and B, together returning the value. If overflow would occur set err$\_$code to EINVAL(22)
\lstinputlisting[language=Python, frame=single]{multiply}
\end{minipage}\\\\

\paragraph*{File Functions}Read, Write, Create, and Filesize are supported\par

\begin{minipage}{\linewidth}
\subparagraph*{read} Read bufsize bytes from file into buffer starting at the offset and return the number of bytes read if there was no error, -1 otherwise.
\lstinputlisting[language=Python, frame=single]{read_file}
\end{minipage}\\\\

\begin{minipage}{\linewidth}
\subparagraph*{write} Write buffsize bytes from buffer to a file starting at offset and return the number of bytes written if there was no error, -1 otherwise.
\lstinputlisting[language=Python, frame=single]{write_file}
\end{minipage}\\\\

\begin{minipage}{\linewidth}
\subparagraph*{create}Create a new 0 byte file if it does not already exist, returns -1 if an error occurs
\lstinputlisting[language=Python, frame=single]{create_file}
\end{minipage}\\\\

\begin{minipage}{\linewidth}
\subparagraph*{filesize} Returns the size of an existing file, -1 in the event of an error
\lstinputlisting[language=Python, frame=single]{filesize}
\end{minipage}\\\\

\end{document}